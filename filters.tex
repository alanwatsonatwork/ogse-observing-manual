\chapter{Filter}

\begin{table}
\centering
\caption{Filters}
\medskip
\label{table:filters}
\footnotesize
\begin{tabular}{lcccccl}
\toprule
Filter&$\lambda_\mathrm{S}$&$\lambda_\mathrm{L}$&$\Delta\lambda$&$\bar\eta$&$n_0$&\multicolumn{1}{c}{Color Term}\\
&nm&nm&nm&&\unit{e/s}\\
\midrule
$r$   & $\approx 590$ & $\approx 690$ & $\approx 100$ & TBD & TBD & $+0.10 \times (g-r)$\\
\bottomrule
\end{tabular}
\end{table}


\section{Filters}

The OGSE filters are summarized in Table~\ref{table:filters}.

The $r$ filter is a Baader red filter with a blue edge at about 590 nm and a red edge at about 690 nm. It is narrower and redder than the standard SDSS/Pan-STARRS $r$ filter. Its redness can be seen by its color term of approximately $The $r$ filter is a Baader red filter with a blue edge at about 590 nm and a red edge at about 690 nm. It is narrower and redder than the standard SDSS/Pan-STARRS $r$ filter. Its redness can be seen by its color term of approximately $+0.10 \times (g-r)$.


\section{Transformations}

Table~\ref{table:filters} shows typical color terms for the transformations from natural magnitudes to the corresponding magnitudes in the Pan-STARRS and Landolt systems. We transform $gri$ to Pan-STARRS $r$, not to Pan-STARRS $w$. The color terms are in the sense
\begin{align}
\mbox{Standard magnitude} = \mbox{OGSE magnitude} + \mbox{color term}.
\end{align}
For most filters, the color terms are small. The exception is $B$, for which the color term is $+0.24(B-V)$.

\section{System Efficiency}

\section{Zero-Points}

Table~\ref{table:filters} also gives the estimated zero points $n_0$ for each filter, defined by 
\begin{align}
n_0 = \frac{A F_0}{h}\int \frac{\eta}{\lambda}\,d\lambda,
\end{align}
in which $A$ is the area of the telescope (discounting the central obstruction), $h$ is Planck's constant, and $F_0 = 3631~\unit{Jy}$ is the flux density of a source with $\AB = 0$. In practical terms, the zero point $n_0$ is the expected signal in $\unit{e/s}$ from a source with zero AB magnitude at all wavelengths.

Observers are cautioned that these zero points are optimistic, in that they assume fresh aluminum coatings. The M1 and M2 mirrors were coated in June and September 2024, and the reflectivity is expected to decrease by about 10\% per year. Furthermore, clouds can also reduce the observed signal.

NOTE: We are still working to confirm the measured zero points, but at the moment there is no reason to believe that the theoretical estimates are significantly incorrect.
